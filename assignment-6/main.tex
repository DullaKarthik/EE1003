\documentclass[journal]{IEEEtran}
\usepackage[a5paper, margin=10mm, onecolumn]{geometry}
%\usepackage{lmodern} % Ensure lmodern is loaded for pdflatex
\usepackage{tfrupee} % Include tfrupee package

\setlength{\headheight}{1cm} % Set the height of the header box
\setlength{\headsep}{0mm}     % Set the distance between the header box and the top of the text

\usepackage{gvv-book}
\usepackage{gvv}
\usepackage{cite}
\usepackage{amsmath,amssymb,amsfonts,amsthm}
\usepackage{algorithmic}
\usepackage{graphicx}
\usepackage{textcomp}
\usepackage{xcolor}
\usepackage{txfonts}
\usepackage{listings}
\usepackage{enumitem}
\usepackage{mathtools}
\usepackage{gensymb}
\usepackage{comment}
\usepackage[breaklinks=true]{hyperref}
\usepackage{tkz-euclide} 
\usepackage{listings}
% \usepackage{gvv}                                        
\def\inputGnumericTable{}                                 
\usepackage[latin1]{inputenc}                                
\usepackage{color}                                            
\usepackage{array}                                            
\usepackage{longtable}                                       
\usepackage{calc}                                             
\usepackage{multirow}                                         
\usepackage{hhline}                                           
\usepackage{ifthen}                                           
\usepackage{lscape}
\begin{document}

\bibliographystyle{IEEEtran}
\vspace{3cm}

\title{11.16.4.7.4}
\author{EE24BTECH11017 - D.Karthik}
 \maketitle
% \newpage
% \bigskip
{\let\newpage\relax\maketitle}

\renewcommand{\thefigure}{\theenumi}
\renewcommand{\thetable}{\theenumi}
\setlength{\intextsep}{10pt} % Space between text and floats


\numberwithin{equation}{enumi}
\numberwithin{figure}{enumi}
\renewcommand{\thetable}{\theenumi}
\textbf{Problem Statement :}

Given two events \( A \) and \( B \) such that:


   $ P(A) = 0.54, \quad P(B) = 0.69, \quad P(A  B) = 0.35$

Find \( P(B  A^\prime) \).
\section*{Solution Using Boolean Logic}

Given:
\[
P(A) = 0.54, \quad P(B) = 0.69, \quad P(A  B) = 0.35
\]
We need to find:
\[
P(B A^\prime)
\]

\subsection*{Step 1: Boolean Logic Decomposition}

\begin{align}
    A + A &= A\\
    AA &= A \\
    A + A^\prime  &= 1\\
    AA^\prime  &= 0 \\
    AB &= BA\\
    A + B &= B + A\\
    \brak{A + B} + C &= A + \brak{B + C} \\
    \brak{AB}C &= A\brak{BC} \\
    A\brak{B + C} &= AB + AC \\
    A + BC &= \brak{A + B}\brak{A + C} \\
    P\brak{1} &= 1\\
    P\brak{A + B} &= P\brak{A} + P\brak{B}, \text{ if } P\brak{AB} = 0
\end{align}

De Morgan's Theorems:
\begin{align}
    \brak{A + B}^\prime  = A^\prime  B^\prime  \label{eq:dm1} \\
    \brak{AB}^\prime  = A^\prime  + B^\prime  \label{eq:dm2}
\end{align}
Using these axioms,
\begin{align}
B &=\brak{A+A^\prime }B \\
    &= AB + A^\prime  B \label{eq:B}\\
    P\brak{B} &= P\brak{AB} + P\brak{A^\prime B}\\
    P\brak{A^\prime B} &= P\brak{B} - P\brak{AB}\\
    P\brak{A^\prime B} &= 0.69 - 0.35 \\
    P\brak{A^\prime B} &= 0.34
\end{align}



\section{Solution Using Random Variables and Conditional Probability}

Let \( X \) be an indicator random variable such that:

\[
X =
\begin{cases}
1, & \text{if } B \text{ occurs}, \\
0, & \text{otherwise}.
\end{cases}
\]

Using conditional probability,

\[
P(B) = P(B | A) P(A) + P(B | A^\prime) P(A^\prime)
\]

Rearranging to find \( P(B | A^\prime) P(A^\prime) \):

\[
P(B | A^\prime) P(A^\prime) = P(B) - P(B | A) P(A)
\]

We compute \( P(B | A) \):

\[
P(B | A) = \frac{P(A  B)}{P(A)} = \frac{0.35}{0.54} \approx 0.6481
\]

Similarly, we compute \( P(A^\prime) \):

\[
P(A^\prime) = 1 - P(A) = 1 - 0.54 = 0.46
\]

Now, solving for \( P(B | A^\prime) \):

\[
P(B | A^\prime) = \frac{P(B) - P(A  B)}{P(A^\prime)}
\]

\[
= \frac{0.69 - 0.35}{0.46} = \frac{0.34}{0.46} \approx 0.7391
\]

Thus,

\[
P(B \cap A^\prime) = P(B | A^\prime) P(A^\prime) = 0.7391 \times 0.46 = 0.34
\]

\textbf{Monte Carlo Simulation Approach:} We can verify this result by simulating a large number of trials where:
\begin{enumerate}
    \item Generate a random value to determine if \( A \) occurs.
    \item Generate a random value to determine if \( B \) occurs, conditioned on \( A \) or \( A^\prime \).
    \item Count occurrences where \( B \text{ and } A^\prime \) happens.
\end{enumerate}

Running a simulation with one million trials will produce an estimate close to \( 0.34 \), confirming our calculations.


\end{document}